\documentclass[10pt]{article}
\usepackage{amsgen,amsmath,amstext,amsbsy,amsopn,amssymb}
\usepackage{graphicx,color,epsfig}
\usepackage[colorlinks]{hyperref}
\usepackage{yfonts}
\usepackage{float}
\usepackage[english]{babel}
\usepackage[left=1in,right=1in,top=1in,bottom=1in]{geometry} %a piece of the template i believe.  dab



\usepackage{array}
\usepackage{makeidx}
\makeindex

\usepackage[boxed,linesnumbered]{algorithm2e}
\usepackage[format=plain,indention=.5cm,margin=10pt,font=small]{caption}




 
\newcommand{\singlespace}[1]{{\setlength{\baselineskip}{0.6\baselineskip} {#1} \par}}



\title{Paramotopy}
\author{Daniel Brake}
\date{\today}


%%%%%%%%%%%%%%%%%%%%%%%%%%%%%%%%%%%%%%%%%%%%%%%%%%%%%%%%%%%%%%%%%%%%%%%%%

\begin{document}

\pagestyle{plain} 
\pagenumbering{roman} 
\setcounter{page}{1}


\thispagestyle{empty}
\maketitle

\newpage


\singlespace{
\tableofcontents

%\listoffigures
}
\eject
\pagenumbering{arabic} 
\setcounter{page}{1}
\eject




\section{Introduction}

The Paramotopy program is a linux/unix wrapper around Bertini which permits rapid parallel solving of parameterized polynomial systems.  It uses MPI for inter-process communication.

Paramotopy consists of two executable programs: Paramotopy, and Step2, and further depends on having a copy of the parallel version of Bertini.  Paramotopy is called from the command line, and in turn calls Bertini and Step2.

\subsection*{disclaimer}
Paramotopy is offered without warranty, for any purpose. 

\section{Getting Started}


\subsection{Compilation and Installation}

Paramotopy has the following library dependencies: 
\singlespace{
\begin{itemize}
\item tinyxml
\item MPI
\item af
\end{itemize}
}

\section{Input Files}

\section{Options}

Persistent Configuration of the Paramotopy program is maintained through the \texttt{./paramotopy/prefs.xml} file located in the home directory.

\subsection{Parallelism}

\subsection{Bertini}

\section{Troubleshooting}


\singlespace{
\bibliographystyle{ieeetr}
\bibliography{bibliobiblioparama}
}

\singlespace{\printindex}

\end{document}
